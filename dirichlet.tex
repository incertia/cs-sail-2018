\documentclass{article}

\usepackage[nosection]{incertia}

\pagestyle{fancy}
\lhead{Will Song}
\chead{Dirichlet's Theorem}
\rhead{\today}

\begin{document}
\begin{df}
The $n$-th \textbf{cyclotomic polynomial} $\Phi_n$ is defined to be
\[ \Phi_n(X) = \prod_{\mathclap{\substack{\zeta^n = 1 \\ \zeta^j \neq 1 \\ 0 < j
< n}}} (X - \zeta), \]
where $\zeta$ runs over all the primitive $n$-th roots of unity.
\end{df}

\begin{prop}
$\deg \Phi_n = \varphi(n)$.
\end{prop}

\begin{proof}
Trivial.
\end{proof}

\begin{prop}
$X^n - 1 = \prod_{d \mid n} \Phi_d(X)$.
\end{prop}

\begin{proof}
This is a simple exercise in root counting. Recall that $X^n - 1 =
\prod_{\zeta^n = 1} (X - \zeta)$. For each $\zeta$ such that $\zeta^n = 1$, we
can define $\ord(\zeta)$ as the smallest $k$ such that $\zeta^k = 1$. Notice
that $\ord(\zeta) \mid n$. The factor $X - \zeta$ occurs in precisely
$\Phi_{\ord(\zeta)}(X)$ and nothing else.
\end{proof}

\begin{cor}
$n = \sum_{d \mid n} \varphi(d)$.
\end{cor}

\begin{proof}
Trivial. Just add the degrees.
\end{proof}

\begin{prop}
$\Phi_n(X) \in \ZZ[X]$.
\end{prop}

\begin{proof}
We proceed with induction. The base case $\Phi_1(X) = X - 1$ is clearly true.

Now suppose for all $k < n$ we have that $\Phi_k(X) \in \ZZ[X]$. Define
\[ P_n(X) = \prod_{\mathclap{\substack{d \mid n \\ d \neq n}}} \Phi_d(X). \]
Clearly $P_n$ is monic. We can then use the division algorithm to find
integer polynomials $Q(X), R(X)$ such that
\[ X^n - 1 = P_n(X) Q(X) + R(X) \]
and $\deg R < \deg P_n = n - \varphi(n)$ or $R(X) = 0$. By plugging in the roots
of $P_n$, it is easy to see that $R(X)$ must have at least $n - \varphi(n)$
roots, so if $R$ is non-zero it must have degree at least $n - \varphi(n)$,
which cannot happen. Thus $R(X) = 0$ and $X^n - 1 = P_n(X) Q(X)$, where $Q(X)
\in \ZZ[X]$, but $X^n - 1 = P_n(X) \Phi_n(X)$, which gives us $Q(X) = \Phi_n(X)$
and $\Phi_n(X) \in \ZZ[X]$.
\end{proof}

\begin{prop}
$\Phi_n(X)$ is irreducible.
\end{prop}

\begin{proof}
We actually don't have enough tools to show this for the general case. It is
easy to show that $\Phi_p(X + 1)$ is irreducible and thus $\Phi_p(X)$ is also
irreducible when $p$ is prime though.

Just compute $\Phi_p(X) = 1 + X + X^2 + \cdots + X^{p - 1}$ and apply Eisenstein
at $X + 1$.
\end{proof}

We now take some inspiration from standard calculus, and state some properties
that we will not prove. Note that these hold in $\RR[X]$ and $\ZZ[X]$, but more
importantly, they are also true in $\FF_p[X]$.

\begin{prop}
$(f(x) + g(x))' = f'(x) + g'(x)$.
\end{prop}

\begin{prop}
$(f(x) g(x))' = f'(x) g(x) + f(x) g'(x)$.
\end{prop}

We can now prove the following statement.

\begin{prop}
Let $P$ be a polynomial over either $\RR[X]$, $\QQ[X]$, $\ZZ[X]$, or $\FF_p[X]$.
Then there exists a non-constant polynomial $m(X)$ such that $m(X)^2 \mid P(X)$
if and only if $\gcd(P(X), P'(X)) \neq 1$.
\end{prop}

\begin{proof}
Suppose there exists such a $m(X)$ such that $m(X)^2 \mid P(X)$. Then $P(X) =
m(X)^2 Q(X)$ and $P'(X) = 2 m(X) m'(X) Q(X) + m(X)^2 Q'(X)$, so $m(X) \mid
\gcd(P(X), P'(X))$.

For the other direction, $\gcd(P(X), P'(X)) \neq 1$ implies that they share a
root $X_0$. We claim that $m(X) = X - X_0$ satisfies the double root property
for $P(X)$. Suppose not. Then $P(X) = m(X) Q(X)$ and $Q(X_0) \neq 0$. Compute
$P'(X) = m'(X) Q(X) + m(X) Q'(X)$. Note that $P'(X_0) = 0$ because $X_0$ is a
root of $P'(X)$, but $P'(X_0) = m'(X_0) Q(X_0) + m(X_0) Q'(X_0) = 1$, which is a
contradiction.
\end{proof}

\begin{cor}
$X^n - 1$ only has roots of multiplicity $1$ in any of the above rings.
\end{cor}

\begin{prop}
Let $m, n$ be two positive integers and $p$ a prime such that $p \nmid mn$. Then
\[ \gcd(\Phi_n(X), \Phi_m(X)) = 1 \]
over $\FF_p[X]$.
\end{prop}

\begin{proof}
By the previous proposition $X^{mn} - 1$ taken in $\FF_p[X]$ has no repeated
factors. Suppose for the sake of contradiction that $\gcd(\Phi_n(X), \Phi_m(X))
= g(X) \neq 1$. Then $g(X)^2 \mid X^{mn} - 1$, which is not true. This is
because $X^{mn} - 1 = \prod_{d \mid mn} \Phi_d(X)$ and $\Phi_n, \Phi_m$ both
show up in this product.
\end{proof}

\begin{cor}
Let $m, n$ be two positive integers and $p$ prime such that $p \nmid mn$. Then
$\Phi_n(x), \Phi_m(x)$ cannot be both divisible by $p$ for the same value of
$x$.
\end{cor}

\begin{proof}
Suppose $\Phi_n(x) \equiv 0 \pmod{p}$ and $\Phi_m(x) \equiv 0 \pmod{p}$. Then $X
- x$ is a factor of $\Phi_n(X)$ and $\Phi_m(X)$ in $\FF_p[X]$, which cannot
happen.
\end{proof}

For a given prime $p$, we define the standard order $\ord_p(a)$ to be the least
positive integer $k$ such that $a^k \equiv 1 \pmod{p}$.

Below we give one of the most important theorems in the theory of cyclotomic
polynomials.

\begin{thm}
Let $p$ be a prime. Then for all positive integers $n$ and integers $a$ such
that $\gcd(n, p) = 1$, we have $p \mid \Phi_n(a) \iff \ord_p(a) = n$.
\end{thm}

\begin{proof}
We proceed via induction on $n$. The base case $n = 1$ is trivial since
$\Phi_1(x) = x - 1$ has a root at $x \equiv 1 \pmod{p}$. Now suppose that the
hypothesis is true for all $k < n$.

Suppose $a$ satisfies $\Phi_n(a) \equiv 0 \pmod{p}$. Recall that $a^n - 1 =
\prod_{d \mid n} \Phi_d(a) \equiv 0 \pmod{p}$, so $a^n \equiv 1 \pmod{p}$. Now
suppose that $\ord_p(a) = k \neq n$. By the induction hypothesis, we know that
$\Phi_k(a) = 0$. We also have $k \mid n$, so $\gcd(k, p) = 1$ implies $p \nmid
kn$, which contradicts proposition 12. Thus $\ord_p(a) = n$.

Now suppose $\ord_p(a) = n$. Then $\prod_{d \mid n} \Phi_d(a) = a^n - 1 \equiv
0 \pmod{p}$. If for some $d < n$ we have $\Phi_d(a) \equiv 0 \pmod{p}$, we have
$\ord_p(a) = d$ by the induction hypothesis so $\prod_{\mathclap{\substack{d
\mid n \\ d \neq n}}} \Phi_d(a) \not \equiv 0 \pmod{p}$. Thus we necessarily
have $\Phi_n(a) \equiv 0 \pmod{p}$.
\end{proof}

We can now finally prove the main theorem.

\begin{thm}[Dirichlet]
Let $a, n$ be two relatively prime positive integers. Then there exist
infinitely many primes $p$ of the form $p \equiv a \pmod{n}$.
\end{thm}

Okay I lied. We actually can't prove that. We can only prove the special case of
$a = 1$.

\begin{cor}[Dirichlet]
For any integer $n$ there exist infinitely many primes of the form $p \equiv 1
\pmod{n}$.
\end{cor}

We first prove the following lemmas.

\begin{lem}
Let $p$ be a prime not dividing $n$. If there exists an integer $a$ such that $p
\mid \Phi_n(a)$, then $p \equiv 1 \pmod{n}$.
\end{lem}

\begin{proof}
By theorem 13, we immediately know that $\ord_p(a) = n$, but this implies $n
\mid p - 1$, or $p - 1 \equiv 0 \pmod{n}$, giving the result.
\end{proof}

\begin{lem}[Schur]
Given an integer polynomial $f$, there exist infinitely many primes $p$ such
that there exists an integer $a_p$ such that $p \mid f(a_p)$.
\end{lem}

\begin{proof}
We can write $f(n) = n g(n) + f(0)$. If $f(0) = 0$, then the result is obvious.
Now suppose $f(0) \neq 0$.

Suppose without loss of generality that $\lim_{n \to \infty} f(n) = +\infty$.
Let $f(0) = c$. That means there exists a sufficiently large $N$ such that $k >
N \implies f(k) > |c|$. Pick any of these $k$. Compute
\[ f(c^2 k!) = c^2 k! g(c^2 k!) + c = c (c k! g(c^2 k!) + 1). \]
Then $c k! g(c^2 k!) + 1$ must be divisible by some prime. It is also clearly
not divisible by any prime $p \leq k$ because $k! \equiv 0 \pmod{p}$. Thus this
is divisible by some prime larger than $k$ so we can just take a sequence of
sufficiently large $k$ to get our infinite primes.
\end{proof}

Now we can prove the corollary.

\begin{proof}
We can use lemma 17 to get infinitely many pairs $p, a_p$ such that $p \mid
\Phi_n(a_p)$, and we immediately get $p \equiv 1 \pmod{n}$ by lemma 16.
\end{proof}

The above method can be strengthened into the following proposition, but we do
not have the tools necessary for proving this, so we will just leave it at that.

\begin{prop}
For all integers $n$ and integers $a$ such that $a^2 \equiv 1 \pmod{n}$, there
are infinitely many primes $p$ of the form $p \equiv a \pmod{n}$.
\end{prop}
\end{document}
